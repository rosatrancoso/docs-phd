\chapter{Abstract}\label{sec:abstract} 

Numerical Weather Prediction is increasingly being applied as a support tool in areas that need weather related products to answer specific needs. The aim of the present work was to explore methodologies using the operational meteorological system running at IST to provide accurate wind power forecasts to the Portuguese Transmission System Operator for a secure and economic integration in the grid, and meteorological warnings to the Civil Protection of Lisbon Municipality for a more efficient response in emergency management. 

Results show that the system is a valuable and reliable tool. Wind power forecasts can significantly be improved at the short range (6 h) by Model Output Statistics and at medium term (3 days) by using the Time Lagged Ensemble. Meteorological warnings are given sufficiently in advance to allow the Civil Protection to monitor a given problematic situation and only take action if the warnings persist with forecast updates, avoiding unnecessary economic expenditures. MM5 forecasts performed slightly better than WRF in precipitation, cold weather and wind while WRF was markedly superior in warm weather.

This research allowed concluding that the accuracy of the IST system is comparable to other state of the art systems and the developed methodologies allow significant improvements.


\vspace{1.5cm}%

%\textbf{Keywords:} forecasts, modelling, wind power, grid utility, warnings, civil protection, meteorology, upwelling, verification, statistics
\textbf{Keywords:} forecasts, modelling, wind power, warnings, meteorology,verification
