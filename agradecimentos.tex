\chapter{Acknowledgements}\label{sec:agradec} 

I would like to thank my supervisors for the support and orientation given throughout these years. To Prof. José Delgado Domingos, a special thanks for investing in my learning process and personal growth, and to Prof. João Corte-Real for the insightful discussions and sharing the passion of meteorology.

I would also like to thank Prof. Ramiro Neves for the precious suggestions and profitable discussions, for giving me a space, and for never losing interest.

I am grateful to Prof. José Borges for the interest shown and guidance provided.

This work would have not been possible without the collaboration of Eng. Rui Pestana, from Rede Elétrica Nacional, and Geographer Maria João Telhado from Serviço Municipal de Proteção Civil de Lisboa (SMPC). A special thanks to her and to Eng. Vítor Vieira and Eng. Luísa Coelho, also from SPMC, for their strong support.

I thank Rodrigo Fernandes, Luís Fernandes and Guillaume Riflet from MARETEC for the offshore SST data and operational modelling insights.

I am also grateful to the open source community that supports the meteorological models WRF and MM5 and the graphical and statistical software “The R Project for Statistical Computing”.

In a more personal note, I would like to thank all in the MARETEC ``gang'' for the stimulating working environment, for being comprehensive in times of need and for continuously ``blaming me that it's raining''. To Jorge Palma, João Rodrigues, Sónia Barbosa e Cristina Marta-Pedroso a special thanks for making my work their preoccupation.

To my friends, particularly Inês Costa, Vital Teresa, Sibila Sousa, Alexandre Flor, Eduardo Flor, Carla Lopes, Sofia Gomes, Rui Coelho, Joana Malveiro, João Azevedo, Ana Ribeiro and Susana Mendes for tirelessly being there and making me smile.

And finally, but most important, to my parents and sisters, for their unconditional support, for always believing in me and, in a simple word, for everything.

This work was funded by FCT – Fundação para a Ciência e Tecnologia under the grant SFRH / BD / 17957 / 2004.
