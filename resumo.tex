\chapter{Resumo}\label{sec:resumo} 

A Previsão Numérica do Tempo tem vindo cada vez mais a ser aplicada na resposta a questões específicas com valor económico. O objectivo do presente trabalho foi explorar metodologias com base no sistema operacional de previsão meteorológica implementado no IST para a previsão de energia eólica na perspectiva do Operador do Sistema para uma integração segura e económica na rede, e produção de avisos meteorológicas para a Proteção Civil Municipal de Lisboa no intuito de melhorar a resposta na gestão da emergência.

Os resultados mostram que o sistema é uma ferramenta útil e fiável. As previsões de energia eólica podem ser significativamente melhoradas a curto prazo (6 h) com pós-processamento estatístico, e a médio prazo (3 dias) pela previsão por conjuntos desfasados no tempo. Os avisos meteorológicos são dados com antecipação suficiente permitindo à Proteção Civil monitorizar situações potencialmente problemáticas, agindo caso os avisos persistam com a atualização das previsões, evitando assim despesas desnecessárias. As previsões do MM5 são ligeiramente me-lhores para precipitação, tempo frio e vento sendo as do WRF marcadamente melhores para tempo quente.

Em termos gerais, pode-se concluir que as previsões do IST estão ao nível do estado da arte e as metodologias desenvolvidas permitem melhorias significativas.

\vspace{1.5cm}%

%\textbf{Palavras chave:} previsão, modelação, energia eólica, rede elétrica, avisos meteorológicos, proteção civil, meteorologia, afloramento costeiro, verificação, estatística
\textbf{Palavras chave:} previsão, modelação, energia eólica, avisos, meteorologia, verificação
