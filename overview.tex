\chapter{Overview}\label{sec:overview} 

This introductory section contains an outline of the thesis structure, presenting a brief overview of each chapter and our respective publications.

\subsubsection*{Chapter 1}

The scope of the thesis is presented, along with a description of the numerical weather prediction system running at IST and the available forecasts for Continental Portugal. These will be used as a support tool in subsequent chapters.

\subsubsection*{Chapter 2}

This chapter presents the developed methods to produce wind power forecasts for the Portuguese grid utility, namely predictability studies comparing wind speed and power \citep{AME2006, ENER2006} and methods to improve direct model output, in the short term, using different schemes of Model Output Statistics, and in the medium term, using the Time Lagged Ensemble \citep{Aveiro2007, EMS2007, EWEC2008}. A study to verify the influence of weather regimes in forecast uncertainty is also included \citep{EWEC2008}.

\subsubsection*{Chapter 3}

A modelling experiment with realistic Sea Surface Temperature is presented to verify the influence of upwelling in offshore wind forecasts in the Portuguese Coast \citep{EOW2009}.

\subsubsection*{Chapter 4}

An early warning system for meteorological risk in Lisbon is presented, with meteorological forecasts and warnings given by the IST system. MM5 and WRF forecasts are compared against observations and the error in issued warnings is quantified through categorical verification statistics \citep{ISCRAM2011}.

\subsubsection*{Chapter 5}

Here, conclusions are presented, pointing out the main achievements, limitations and future developments.